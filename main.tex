\documentclass[a4paper,10pt,oneside,twocolumn,notitlepage,final]{jarticle}
\usepackage{ss20_UTF-8}

\usepackage[dvipdfmx]{graphicx}

\author{谷口 暁星(名古屋大学大学院 理学研究科)}
\title{データ科学的装置開発によるサブミリ波分光観測の高感度化}

\begin{document}

\abst{
本講演では、観測データが持つ統計的な性質に注目して観測や解析の方法をデザインし直すことで、ソフトウェアの面からミリ波・サブミリ波分光観測の感度向上を目指す「データ科学的装置開発」の概要と成果を発表する。
ミリ波・サブミリ波は、ダストによる減光を受けずに分子・原子ガスの分光観測が可能な波長帯として、近傍から遠方宇宙にわたる星形成活動の研究に使われている。
ALMA望遠鏡(干渉計)による高感度・高空間分解能を生かした個別天体の観測成果は今や枚挙にいとまがない。
一方、空間・周波数の3次元空間の大規模探査による普遍的な星形成史の解明のためには、干渉計では難しい広視野観測が可能な単一望遠鏡(単一鏡)の役割が重要となる。
現在、口径50m級の次世代単一鏡計画(LST・AtLASTなど)や、超広帯域の分光装置開発(DESHIMAなど)が精力的に進められている。
ところが、単一鏡の分光観測やデータ解析の方法論は、驚くべきことに電波天文学が誕生した半世紀前からほとんど検討・改善がなされていない。
特に、同波長帯では地球大気の熱放射の除去が問題となるが、従来の除去方法を使った観測では、装置が本来達成可能な感度を未だ実現できずにいる。
そこで我々は、単一鏡分光観測の時間×周波数の行列データにおいて、大気放射が周波数方向に相関する性質(低ランク性)や、天体信号がデータに占める割合が小さい性質(スパース性)を積極的に利用して大気放射の除去手法を開発した。
低ランク性を利用した周波数変調観測では、従来のポジションスイッチ観測に比べて1/3の観測時間で同等の感度を達成できることを示した(Taniguchi et al. 2020)。
さらに、スパース性を利用した解析では、従来観測の取得済みデータの感度をも$\sqrt{2}$倍向上できることを示した。
本講演では、実際の観測データを使った実用例とともに、将来計画における応用も紹介する。
}

\section{Introduction}

\section{Methods}

\section{Results}

\section{Discussion}

\section{Conclusions}

\small
\begin{thebibliography}{99}
\end{thebibliography}
\end{document}
