\documentclass[a4paper,10pt,oneside,twocolumn,notitlepage,final]{jarticle}
\usepackage{ss20_UTF-8}

\usepackage[dvipdfmx]{graphicx}
\newcommand{\subrm}[1]{_{\mathrm{#1}}}
\newcommand{\suprm}[1]{^{\mathrm{#1}}}

\author{谷口 暁星(名古屋大学大学院 理学研究科)}
\title{データ科学的装置開発によるサブミリ波分光観測の高感度化}

\begin{document}

\abst{%
本講演では、観測データが持つ統計的な性質に注目して観測や解析の方法をデザインし直すことで、ソフトウェアの面からミリ波・サブミリ波分光観測の感度向上を目指す「データ科学的装置開発」の概要と成果を発表する。
ミリ波・サブミリ波は、ダストによる減光を受けずに分子・原子ガスの分光観測が可能な波長帯として、近傍から遠方宇宙にわたる星形成活動の研究に使われている。
ALMA望遠鏡(干渉計)による高感度・高空間分解能を生かした個別天体の観測成果は今や枚挙にいとまがない。
一方、空間・周波数の3次元空間の大規模探査による普遍的な星形成史の解明のためには、干渉計では難しい広視野観測が可能な単一望遠鏡(単一鏡)の役割が重要となる。
現在、口径50m級の次世代単一鏡計画(LST・AtLASTなど)や、超広帯域の分光装置開発(DESHIMAなど)が精力的に進められている。
ところが、単一鏡の分光観測やデータ解析の方法論は、驚くべきことに電波天文学が誕生した半世紀前からほとんど検討・改善がなされていない。
特に、同波長帯では地球大気の熱放射の除去が問題となるが、従来の除去方法を使った観測では、装置が本来達成可能な感度を未だ実現できずにいる。
そこで我々は、単一鏡分光観測の時間×周波数の行列データにおいて、大気放射が周波数方向に相関する性質(低ランク性)や、天体信号がデータに占める割合が小さい性質(スパース性)を利用した大気放射の除去手法を開発した。
低ランク性を利用した周波数変調観測では、従来のポジションスイッチ観測に比べて1/3の観測時間で同等の感度を達成できることを示した。
さらに、スパース性を利用した解析では、従来観測の取得済みデータの感度をも$\sqrt{2}$倍向上できることを示した。
}

\section{Introduction}

138億年の宇宙史において星形成活動がどのように始まり、変化を経て現在に至ったのか。
これらを解き明かすことは銀河の形成進化を理解する上で極めて重要である。
ところが、$z\gtrsim4$の初期宇宙における星形成活動の役割は、多波長観測により宇宙の星形成率密度の変遷が測定される現在でも依然として解明されていない。
この原因として、赤方偏移の決定精度が悪いこと、銀河の統計的サンプル数が圧倒的に不足していることが挙げられる。

こうした背景から、分子や電離原子の輝線を使った分光学的な赤方偏移決定に基づく、ダストの吸収を受けにくいサブミリ波による初期宇宙の分光探査が有力視されている。
明るい銀河に対しては、ALMA望遠鏡による輝線検出による分光探査の方法論が確立しつつある。
それでも干渉計であるALMA望遠鏡では視野や分光帯域が狭く一般的な明るさの銀河の観測は難しいため、星形成活動を明らかにするために十分な数の測定は得られない。
そこで、空間・周波数の3次元の宇宙論的体積を無バイアスに分光探査する、広視野のアンテナと広帯域の分光撮像装置を搭載した口径50mクラスの地上大型サブミリ波単一望遠鏡(単一鏡)が現在計画されている。

多くの開発研究を通して、ハードウェアの技術的課題は実用化の目途が立っている。
一方、ソフトウェア(観測・解析手法)の課題はまだ十分には解決の目途が立っていない。
地上のサブミリ波望遠鏡による天体検出は、同波長帯で非常に明るい地球大気の放射との闘いである。
特に主成分である水蒸気は風に乗って空間的・時間的に変動するため、観測時・解析時の大気のモデリングは必須であり、不正確なモデルは観測装置が持つ感度以上に輝線の検出を制限する原因にすらなる。
にもかかわらず、大気のモデリングの方法論は驚くべきことに日本の電波天文学が誕生した半世紀前からほとんど改善されておらず、ハードウェアに合わせたソフトウェアの不在が分光探査の最大のボトルネックとなっているのである。


本講演では、観測データが持つ統計的な性質に注目して観測や解析の方法をデザインし直すことで、ソフトウェアの面からミリ波・サブミリ波分光観測の感度向上を目指す「データ科学的装置開発」の概要と成果を発表する。
\ref{s:methods}節では、現在最も一般的に用いられる観測手法であるポジションスイッチ観測の問題点とデータ科学的方法に基づく解決策を提案する。
\ref{s:results}節では、データ科学的方法に基づく2種類の観測・解析手法を紹介し、実観測データへ適用することで感度の向上を実証する。
最後に、\ref{s:discussion}節で手法の応用範囲や将来計画に対する展望を議論する。

\section{Methods}
\label{s:methods}

\subsection{ポジションスイッチ観測と問題点}

ポジションスイッチ観測は、観測天体の座標(いわゆるON点)と別の座標(いわゆるOFF点)を交互に観測し、得られたスペクトル同士を減算することで、観測信号から地球大気放射を除去する手法である。
観測感度$\Delta T$は以下の通りに表される。
\begin{equation}
    \Delta T
    \simeq\frac{\sqrt{2}\,T\subrm{sys}}{\sqrt{\Delta\nu\, t\subrm{total}\, \eta\subrm{obs}}}.
    \label{eq:psw-sensitivity}
\end{equation}
ここで$T\subrm{sys}$はシステム雑音温度、$\Delta\nu$は分光計の周波数分解能、$t\subrm{total}$は総観測時間、$\eta\subrm{obs}$は総観測時間に対するON点観測時間の割合(観測効率)である。
ポジションスイッチ観測は観測・解析手法として簡便なため、多くの単一鏡で天体の一点観測に現在も用いられている。
また、単一鏡によるマッピング観測でも同様にスペクトル同士の減算による大気除去が行われている。
一方、以下に列挙した問題点により、観測装置が本来持つ感度よりも観測感度が大きく低下することが知られている。

\begin{itemize}
    \item ON・OFF点のスペクトルは共に熱雑音を含むため、減算によってノイズレベルが$\sqrt{2}$倍悪化する(そのため式\ref{eq:psw-sensitivity}の分子に$\sqrt{2}$が現れる)
    \item 時間・空間的に異なる大気成分同士の減算を行うため、大気の変動をキャンセルできなかった場合、スペクトルのベースラインがうねる(実効的なノイズレベルを悪化させる)
    \item ON・OFF点を同じ割合で観測する必要があるため、観測効率が低い(望遠鏡の2点間の移動時間も考慮すると、$\eta\subrm{obs}<0.5$となる)
\end{itemize}

\section{Results}
\label{s:results}


\section{Discussion}
\label{s:discussion}


\section{Conclusions}

\small
\begin{thebibliography}{99}
\end{thebibliography}
\end{document}
